\documentclass[a4paper, 10pt, DIV=8]{scrartcl}

% ----- typesetting area ----- %
\newlength{\afourwiddth}
\setlength{\afourwiddth}{210mm}
\newlength{\afourheigght}
\setlength{\afourheigght}{297mm}
% simulate the typesetting area of DIV=8 of a4paper
% (i.e. 5/8 of the height and width of the paper are used for typesetting)
% in a letterpaper this is purely for printing purposes
% also use a large right margin for nicer "todo" notes
% also for some reason the name `afourwidth` is already reserved, so I added a misspelling.
\usepackage
  [ letterpaper
  , textwidth = .625\afourwiddth
  , textheight = .625\afourheigght
  , right = 65mm
  ]{geometry}


% ----- language ----- %
\usepackage[english]{babel}

% ----- basic packages ----- %
\usepackage{hyperref}
\usepackage{amsmath}
\usepackage{amsthm}
\usepackage{thm-restate}
\usepackage{amssymb}
\usepackage{cleveref}
\usepackage{mathtools}
\usepackage{microtype}
\usepackage{graphicx}
\usepackage{enumitem}
\usepackage[x11names]{xcolor}
\usepackage{mdframed}

% ----- font ----- %
\usepackage[T1]{fontenc}
\usepackage{mlmodern}
% looks like the default font but slightly bolder, better for on-screen reading imo

% ----- math fonts: calligraphic, double-stroked and fraktur ----- %
\usepackage[ cal=txupr  , calscaled=1.0
           , bb=px      , bbscaled=1.0
           , frak=euler , frakscaled=1.0
           ]{mathalfa}

% ----- set up AMS Theorem ----- %
\addto\captionsenglish{\renewcommand{\proofname}{\normalfont\textsc{Proof.}}}
% makes head of proof environments smallcaps

\def\thmPunct{.}
\def\thmSpace{.7em}
\newtheoremstyle{atom}{}{}{\upshape}{0pt}{}{\thmPunct}{\thmSpace}%
{{\bfseries\thmname{#1 }\thmnumber{#2}}{\bfseries\sffamily\thmnote{ \color{SlateBlue4}$ \blacktriangleright $ #3}}}

\theoremstyle{plain}
\newtheorem{theorem}{Theorem}[section]
\newtheorem{lem}[theorem]{Lemma}
\newtheorem{prop}[theorem]{Proposition}
\newtheorem{cor}[theorem]{Corollary}

\theoremstyle{atom}
\newtheorem{defn}[theorem]{Definition}
\newtheorem{nota}[theorem]{Notation}
\newtheorem{warn}[theorem]{Warning}

\newtheorem{exmp}[theorem]{Example}
\newtheorem{constr}[theorem]{Construction}
\newtheorem{rem}[theorem]{Remark}
\newtheorem{idea}[theorem]{Idea}

\newtheorem{question}[theorem]{Question}
\newtheorem{conj}[theorem]{Conjecture}
\newtheorem{problem}[theorem]{Problem}
\newtheorem{garbage}[theorem]{Garbage}
\newtheorem{goal}[theorem]{Goal}

\newtheorem{atom}[theorem]{}

% ----- common commands ----- %
\newcommand*{\mb}{\mathbf}
\newcommand*{\mc}{\mathcal}
\newcommand*{\mf}{\mathfrak}
\newcommand*{\ms}{\mathsf}
\newcommand*{\mbb}{\mathbb}
\newcommand*{\N}{\mb N}
\newcommand*{\Z}{\mb Z}
\newcommand*{\set}[2]{\left\lbrace #1 \mid #2 \right\rbrace}
% use syntax \set{#X}{#condition} to generate set that looks like {x : condition(x)}.

% ----- make \emph boldface ----- %
%\let\emph\relax
%\DeclareTextFontCommand{\emph}{\bfseries\em}

% ----- Blank footnote for footnote title ----- %
\makeatletter
\def\blfootnote{\xdef\@thefnmark{}\@footnotetext}
\makeatother

% ----- USE TIKZ ----- %
\usepackage{tikz}
\usetikzlibrary{cd}     % tikzcd -- package for commutative diagrams
\usetikzlibrary{babel}  % tikz does not interact well with non-english babel
\usetikzlibrary{arrows.meta,decorations.pathmorphing,backgrounds,positioning,fit,petri}
% ^-- general imports, allows most common drawings in tikz
\tikzcdset
  { arrow style = tikz
  , diagrams = {> = {Straight Barb[scale = 0.8]}}
  } % change the style of arrows to look good with most fonts.


% ----- to do notes ----- %
\usepackage{todonotes}
\setlength{\marginparwidth}{5cm}
\newcommand{\todoi}[1]{%
  \todo[inline,linecolor=black, backgroundcolor=gray!30!white, bordercolor=black, textcolor=black]{#1}%
}
\newcommand{\todom}[1]{%
  \todo[linecolor=black, backgroundcolor=gray!30!white, bordercolor=black, textcolor=black]{#1}%
}

% ----- set up bibliography ----- %
\bibliographystyle{plain}

% ----- code listings ----- %
\usepackage[outputdir=build]{minted}
% needs pygmentize, see document

\newminted[code]{text}{autogobble}
\newmintinline[icode]{text}{}
%default environments with no syntax highlighting

% ---- set up author, title and date ----- %
\author{Peter Hanukaev}
\title{Basic \LaTeX{} Template%
  \blfootnote{Version 2023-03-28}
}




\begin{document}

\maketitle

\section{Text}

Lorem ipsum dolor sit amet, consectetur adipiscing elit.
Donec libero justo, volutpat a vestibulum vel, mattis at dolor.
Aenean nec congue leo.
Integer luctus lacinia turpis sit amet maximus.
Donec malesuada feugiat metus nec laoreet.
Pellentesque facilisis imperdiet nisl, ut feugiat ipsum porttitor sed.
Lorem ipsum dolor sit amet, consectetur adipiscing elit.
Etiam vel nisl arcu.
Phasellus bibendum turpis odio.
Integer in pulvinar purus.
Aliquam erat volutpat.
Nullam tempor sem vel ipsum viverra, dignissim fringilla metus dignissim.

Proin dui tellus, tincidunt et nisl nec, commodo auctor sapien.
Maecenas accumsan consectetur velit sit amet imperdiet.
Proin vel pharetra quam.
Etiam a finibus lacus.
Proin convallis tempus orci ut facilisis.
Suspendisse ut neque aliquet, iaculis risus non, scelerisque ex.
Morbi vitae metus ullamcorper, viverra nulla quis, mollis turpis.
Vestibulum quis purus id orci iaculis euismod quis at augue.
Sed consectetur mi urna, quis vehicula turpis finibus sed.
Suspendisse faucibus imperdiet congue.
Pellentesque tristique, ex in mattis efficitur, nulla ipsum lacinia elit, cursus interdum magna risus quis mauris.
Etiam tempor ante a rutrum tincidunt.
Quisque a urna sit amet tellus maximus bibendum.
Nulla dictum convallis interdum.
Duis sed bibendum urna.

In pellentesque urna at nulla feugiat, vitae blandit sem malesuada.
Nulla vel urna tellus.
Aliquam erat volutpat.
Donec ut iaculis metus.
Curabitur at elit vel metus ultrices feugiat.
Nam consequat lacinia magna quis ultricies.
Phasellus neque nunc, laoreet ac laoreet quis, semper sed leo.
Maecenas rutrum convallis rhoncus.
Praesent vitae lacus ut orci semper fringilla nec nec eros.
Etiam tincidunt ligula quis turpis venenatis malesuada.


\section{Code}

To use code, use the file minted.tex and
run latexmk with the \icode@-shell-escape@ option.
Additionally, pygmentyze is required.
To get it, install python pip, then run
\begin{code}
  $ pip install pygments
\end{code}

By default, the minted.tex provides the environment
\icode@code@ for displayed listings and the command \icode@\icode@
for inline listings.
Both of them are treated as plaintext, i.e. do not have any syntax highlighting.
For more information, refer to the minted documentation.


\end{document}
